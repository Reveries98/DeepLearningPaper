%!TEX root = ../main_article.tex

\section{Related Work}

% In this section, we introduce existing code search methods and machine learning debiasing methods. 
% code search methods

Our work is closely related to the following two domains:

\vspace{3pt}
\noindent\textbf{Code Search.} There is a tremendous amount of research on code search. 
Early works adopt traditional information retrieval methods like Boolean Model~\cite{SaltonFW83}, Vector Space Model~\cite{SaltonWY75} 
and Structural Semantic Indexing~\citep{Dumais04} to estimate 
the relevance between the query and a code snippet~\citep{LvZLWZZ15,BajracharyaOL10}.
Recent works adopt deep neural networks to embed query and code into vectors.
Then, the code search task is performed by measuring the similarity 
(e.g., cosine similarity) between vectors.
% \yl{some are bi-encoder architecture, as described here, but many are cross-encoder architecture such as CodeBERT. Should describe them separately.}
Along this direction, various deep learning based code search methods have been proposed, 
including but not limited to 
recurrent neural network (RNN) based approaches~\citep{DeepCS}, 
convolutional neural network (CNN) based approaches~\citep{CQIL, ShuaiX0Y0L20}, 
graph neural network (GNN) based approaches~\citep{WanSSXZ0Y19}
and pre-training approaches~\citep{CodeBERT, GraphCodeBERT, GuoLDW0022}.


\subsection{Code Search}
In this section, we introduce existing code search methods and machine learning debiasing methods. 

The challenge of code search is to effectively measure the 
semantic similarity between natural language queries and program code. 
More specifically, it is to find the semantically best matching 
answer from several candidate codes when facing query statements 
entered by developers. Code search methods can be divided into two categories: 
information retrieval based models and deep learning based models. 
The former is usually based on keyword matching. 

Researchers have proposed code search models with diverse network architecture: 
(1) RNN based models (\citep{DeepCS}). (2) CNN based models (\citep{CQIL, ShuaiX0Y0L20}). (3) PLMs (\citep{CodeBERT, CoCLR, GuoLDW0022}). 
(4) Graph based models (\citep{GraphCodeBERT, GuCM21}).
In all of them, the core idea is to measure the similarity between queries and code. 
Common to these approaches is the conversion of queries and 
codes into high-dimensional embedding vectors.

